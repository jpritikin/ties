\documentclass[12pt]{article}
\usepackage{apacite}
\usepackage{soul}

\begin{document}

\setlength{\parindent}{0cm}
\setlength{\parskip}{1cm plus4mm minus3mm}

\section{Face sheet}

Title of Proposal: \textbf{Mental Silence Disposition Scale}

Dates of Proposed work: Jan 2014 - May 2015

Name: Joshua N. Pritikin

UVa Department: Psychology

UVa Status (job title, student status, etc.: \\
3rd year graduate student in the quantitative area (masters completed)

Other Participants (list name, title and affiliation for each): \\
Karen Schmidt, Professor of Quantitative Psychology

\break

\setlength{\parskip}{1ex plus2mm minus1mm}

\section{Abstract}
% less than 250 words

Not written yet

\section{Goals}

Examine the convergent and discriminant validity of our new mental
silence disposition scale compared with a broad selection of popular
psychological measures.

\section{Plan/method}

Approximately 40 related psychological measures have been identified.
A sample of 1000 participants will each complete this battery of measures.
The administration of many psychological scales on a single occasion
is a taxing experience for participants. 
Therefore, we will compensate participants with \$5 per response.
Once data are obtained,
we are well prepared to perform the requisite data analysis.

\section{Background}

\begin{quotation}
\noindent Because of a limitation in our vocabulary, the West refers to the simple
psychological centering devices, the preliminary steps, as ``meditation.''
We therefore give the same name to the techniques used to produce
meditation as we do to the end state itself. According to the great
meditative traditions, however, the centering techniques are not
meditation. They are simply means toward the goal---which is meditation.
These techniques are therefore more or less interchangeable, and the
advanced practitioner will eventually discard all of them when he can
achieve meditation directly. \cite[pp.~8--9]{carrington1977}
\end{quotation}

We agree with Carrington, but many
researchers have struggled to agree on a specific
definition of meditation (\citeNP[p.~1]{ospina2007};
\citeNP[pp.~601--602]{shapiro2009}).
In two recent reviews of the state of the research on meditation/mindfulness,
the possibility of mental silence or thoughtless awareness was not
even acknowledged (\citeNP{baer2011}; \citeNP{shapiro2009}).
Mental silence did not even merit a glossary entry in reference works on
positive psychology and the psychology of religion
(\citeNP{pargament1997}; \citeNP{peterson2004};
\citeNP{sheldon2011}; \citeNP{snyder2009}).

\textbf{The question remains unresolved: is \emph{complete mental silence} a
potential psychological state?}

In our review of scales related to spiritual and religious experience,
we found no focused attempt to assess
``mental silence'' or ``thoughtless awareness'' (\citeNP{fetzer1999};
\citeNP{hill1999}; \citeNP{monod2011}).
Therefore, we set out to develop a new measure.
Preliminary results were presented at the conference Spirituality in the
21st Century \cite{pritikin2013}. %TODO update citation
The measure was substantially revised in early 2013.
We are currently conducting intervention studies in Australia in
collaboration with Ramesh Manocha and as part of the Buddhist
Meditation class offered by the Religious Studies department at UVa.
To examine concurrent validity, in addition to the mental silence scale,
we administer the State-Trait Anxiety
Inventory, Reflection-Rumination scale, and Pittsburgh Sleep Quality
Index \cite{buysse1989,spielberger1983,trapnell1999}.

\section{CSC Justification}

We bring our expertise with quantitative methods to examine
an intriguing, under-researched component of contemplative practices:
mental silence.

\begin{quotation}
\noindent A meditation session will often follow a certain progression. Starting
with an active type of thought, it may move toward more quiet types of
thinking, and sometimes this process leads to a state where no
thinking seems to occur at all. In mantra meditation, the mantra may
become increasingly soft and indistinct as the meditation session
continues, until, no longer needed, it gives way to profound quiet. In
breathing meditation, awareness of the breath may recede until it
becomes almost imperceptible and silence is the all-encompassing
experience.  According to the meditative tradition, at this point the
mind is not focused on either thought or image, but is fully aware,
conscious. The mind is said to be alert without having any
\emph{object} of alertness.
\cite[p.~91]{carrington1977}
\end{quotation}

The Contemplative Sciences Center should fund this project because
a self-report measure for mental silence will make it easier to design
and perform research in contemplative sciences.
This in turn will help foster reciprocal
partnerships between scientists, humanistic scholars, and practitioners.
We already have some means of assessing the effectiveness of interventions
designed to foster peace of mind.
A measure of mental silence could serve as another important yardstick.

By supporting this research, UVa will benefit by being at the
epicenter of a new, important self-report scale. UVa will gain reputation as a
premiere institution for the study of contemplative sciences and this
will attract more funding to the university.

\section{Dissemination plans}

We believe that a package of intervention studies and
correlations with a broad selection of related self-report scales is sufficient to
publish our mental silence disposition scale in a leading
psychological journal such as JPSP.

\section{Plan for obtaining additional funding}

We have no plans for additional funding.

\section{Budget}

License fees for measures and scales -- \$?

Incentives for participation \$5000

may include applicant effort, support personnel, consulting services,
supplies, incentives for student/subject participation, cost-share,
and so forth. In addition to the tabular budget, please include a
narrative justification.

\bibliographystyle{apacite}
\bibliography{jpritikin}

\end{document}
