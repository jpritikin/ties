
\documentclass[10pt,utf8x]{beamer}
\usefonttheme{serif}

\mode<presentation>
{
  \usetheme{CVille}
  \usecolortheme{cavs}
  \setbeamercovered{transparent}
  % or whatever (possibly just delete it)
}


\usepackage[english]{babel} % or whatever
\usepackage{multimedia}
\usepackage{apacite}
\usepackage{color}
\usepackage{listings}
\lstset{
language=R,
basicstyle=\scriptsize\ttfamily,
numbers=left,
numberstyle=\ttfamily\color{gray}\footnotesize}
\usepackage{tikz}
\usepackage{soul}

\newlength{\figurewidth} 
\figurewidth \textwidth  % This is for rectangular graphs 
\newlength{\figurewidthB} 
\figurewidthB .7\textwidth  % This is for square graphs 

\title{Is there a link between flow and meditation?}

\author{Joshua N. Pritikin}

\institute[University of Virginia] % (optional, but mostly needed)
{
  Department of Psychology\\
  University of Virginia
}

\date[DADA] % (optional, should be abbreviation of conference name)
{\footnotesize 19 Apr 2012}

\pgfdeclareimage[height=0.4cm]{university-logo}{logo.png}
\logo{\pgfuseimage{university-logo}}

\usepackage{Sweave}
\begin{document}

\begin{frame}
  \titlepage
\end{frame}

\setlength{\parskip}{2ex}

\section{Flow}
\begin{frame}
\frametitle{Flow a.k.a. Optimal Experience}
Flow is an intrinsically rewarding experience in which a person
is absorbed in the optimal performance of an activity. 

Flow can occur in diverse contexts
\cite{csikszentmihalyi1991}\footnote{Csikszentmihalyi is pronounced ``C-Z''}
\begin{itemize}
\item Music performance
\item Reading
\item Writing
\item Mountain climbing
\item Japanese motorcycle gangs
\item etc
\end{itemize}
\end{frame}

\begin{frame}
\begin{figure}[tp]
{\scriptsize
\begin{tikzpicture}[>=latex,line join=bevel,scale=.7]
  \pgfsetlinewidth{1bp}
\input{gen/flow.tex}
\end{tikzpicture}
}
\label{fig:teleonomy-of-self}
\end{figure}
\end{frame}

\begin{frame}
\citeA[p.~103]{csikszentmihalyi1991} suggested
that yoga is ``one of the oldest and most systematic methods of
producing the flow experience,''
in the sense of facilitating flow in other activities besides yoga/meditation
(cf. \citeNP[p.~31]{csikszentmihalyi1988}).

There is little outward commonality between stereotypical yogic
practices and the diverse activities in which flow has been found.

What aspect of yoga/meditation might facilitate flow?
\end{frame}

\begin{frame}
\frametitle{\textcolor{red}{Meditation} means two different things!}

\begin{quotation}
Because of a limitation in our vocabulary, the West refers to the simple
psychological centering devices, the preliminary steps, as ``meditation.''
We therefore give the same name to the techniques used to produce
meditation as we do to the end state itself. According to the great
meditative traditions, however, the centering techniques are not
meditation. They are simply means toward the goal---which is meditation.
These techniques are therefore more or less interchangeable, and the
advanced practitioner will eventually discard all of them when he can
achieve meditation directly. \cite[pp.~8--9]{carrington1977}
\end{quotation}

Hence, \emph{meditation} means:
\begin{itemize}
\item Psychological centering devices
\item Experience of complete mental silence or thoughtless awareness
\end{itemize}
\end{frame}  

\begin{frame}  
Psychological centering devices are a dime a dozen (mantra, breathing focus,
visual focus, mindfulness, etc).
Of interest here is complete mental silence.

The notion of thoughtless awareness is found in widely translated,
centuries old
books such as the Mahabharat, Upanishads, Yoga Sutras of Patanjali,
and Gyaneshwari \cite[pp.~93--94]{manocha2009}.

Complete mental silence is an experience in which one is
no longer is thinking and
no longer feels the compulsion to continuously engage in thought.
The experience occurs in a normal waking state;
thoughtless awareness is not similar to sleeping or unconsciousness.
\end{frame}

\begin{frame}
Can thoughtless awareness be a component of or itself a flow experience? 

\begin{itemize}
\item \textcolor{red}{Challenge/Skill} A naive thought suppression approach fails \cite{wegner2003}.
However, \citeA{manocha2000} reported that participants in his
Meditation Research Program ``consistently describe the
ability to achieve this experience'' (p.~1137).
Perhaps with the appropriate training,
the challenge of thoughtless awareness can be made navigable.

\item \textcolor{red}{Clear goal/Immediate feedback} Yes

\item \textcolor{red}{Concentration} Without engrossment
in thoughtless awareness, thinking is bound to resume.

\end{itemize}

Can meet the basic conditions of flow experience but ... ?
\end{frame}

\begin{frame}
\frametitle{Thoughtless awareness has not been studied}

\begin{itemize}
\item PsycNET searches for ``thoughtless awareness'' or ``mental silence''
turn up nothing.
\item
In a collection of 13 scales related to spiritual and religious experience,
no attempt was made to measure ``thoughtless awareness'' \cite{fetzer1999}.
\item
In two recent reviews of the state of the research on meditation/mindfulness,
the possibility of thoughtless awareness or mental silence was not
even acknowledged (\citeNP{shapiro2009}; \citeNP{baer2011}).
\end{itemize}

(continued on the next slide \dots)
\end{frame}

\begin{frame}
\frametitle{Thoughtless awareness has not been studied}
\begin{itemize}
\item
No glossary entry found in reference works on
positive psychology and the psychology of religion
(\citeNP{pargament1997}; \citeNP{peterson2004}; \citeNP{snyder2009}; \citeNP{sheldon2011}).
\item In comparison, extrasensory perception has attracted 2-3 orders of
magnitude more scientific studies \cite{lau2004}.
\end{itemize}

A potentially vital component of flow has not received
any psychological scrutiny.
\end{frame}

\section{Survey}

\begin{frame}
\frametitle{Let's conduct a survey!}

Some hypotheses:
\begin{itemize}
\item Intuitive or accidental thoughtless awareness is not uncommon.
\item Thoughtless awareness occurs outside the context of meditation.
\item Flow and complete mental silence will be judged as similar relative
  to some other obvious comparisons.
\item Differential item functioning (DIF):
Has received any training in meditation? By education?
Trying to intend complete mental silence or not?
\end{itemize}
\end{frame}

\begin{frame}
\frametitle{Let's conduct a survey!}
Questionnaire sections:
\begin{itemize}
\item Biographical information
\item Familiarity with complete mental silence (a.k.a. thoughtless awareness)
\item Complete mental silence in context
\item How is thoughtless awareness related to flow
\item Supplemental questions\footnote{Omitted in this report.}
\end{itemize}
\end{frame}

\begin{frame}
  \frametitle{Biographical information}
Age?

Sex? Male/Female

Highest level of education attained?
\begin{itemize}
\item High school
\item College degree
\item Graduate degree
\end{itemize}

What is your current profession / occupation? (free text)

Have you pursued any training in meditation? Yes/No

If so, what kind of meditation do you practice currently? (free text)
\end{frame}

\begin{frame}
  \frametitle{Familiarity with complete mental silence}

Of interest are your recent experiences of complete mental silence
(within the last year or so).

On a 3-point Likert scale from \emph{agree} to \emph{disagree}, rate:

\begin{itemize}
\item The notion of complete mental silence does not make sense
\item I doubt that complete mental silence is possible for anybody
\item I doubt that complete mental silence is possible for myself
\item I have experienced complete mental silence accidentally
\item I have intentionally experienced complete mental silence
\item I am certain that complete mental silence is possible for anybody
\end{itemize}

\end{frame}

\begin{frame}
  \frametitle{Context of complete mental silence}

On a 5-point Likert scale from \emph{important} to \emph{not important}, rate:

\begin{itemize}
\item Is complete mental silence an important component of religion?
\item ... of problem solving?
\item ... of spiritual practice?
\item ... of relaxation?
\item ... of daydreaming?
\item ... of meditation?
\item ... of physical exercise?
\end{itemize}

\end{frame}

\begin{frame}
  \frametitle{Relationship to flow}

\emph{Needs more brainstorming \dots}

\begin{itemize}
\item Would complete mental silence help prepare you for flow?
\item Would flow help prepare you for complete mental silence?
\item How similar are complete mental silence and flow?
\item How compatible are complete mental silence and flow?
\item How similar are blue and purple?
\item\emph{What else?}
\end{itemize}

\end{frame}

\section{IRT}
\begin{frame}
  \frametitle{Item Response Theory -- Partial Credit Model}

The Generalized Partial Credit Model (GPCM) model can be expressed in terms
of the unconditional probability of each
response $0, 1, \dots, m_i$ of person $j$'s attempt at item $i$. That is,

\begin{displaymath}
\sum_{h=0}^{m_i}P_{ijh} = 1.
\end{displaymath}

(continued on the next slide \dots)
\end{frame}

\begin{frame}
  \frametitle{Item Response Theory -- Partial Credit Model}

The probability of person $j$ scoring $x$ on item $i$ is

\begin{displaymath}
\forall x \in \{0,1,\dots,m_i\}, \quad
P_{ijx} =
\frac{\exp\left[\ \displaystyle\sum_{k=0}^{x} \alpha_{ik}(\theta_j - \delta_{ik})\right]}
{\displaystyle\sum_{h=0}^{m_j}
\left[\exp\ \displaystyle\sum_{k=0}^{h}\alpha_{ik}(\theta_j - \delta_{ik})\right]}
\end{displaymath}

where $\theta_j$ is person $j$'s trait level,
$\delta_{ik}$ is the difficulty of the item $i$ at category $k$,
$\alpha_{ik}$ is the discrimination of the item $i$ at category $k$, and \dots
\end{frame}

\begin{frame}
  \frametitle{Item Response Theory -- Partial Credit Model}

\dots summing across all persons $j$,
item parameters $\alpha$ and $\delta$ are centered at zero

\begin{displaymath}
\sum_j\alpha_{ij}(\theta - \delta_{ij}) \equiv 0.
\end{displaymath}

Unfortunately, no \textcolor{red}{R} implementation of GPCM is
working well enough yet.
Data were analyzed using the \textcolor{red}{eRm} package \cite{erm2011}, fixing $\alpha=1$.
This is parameterization of GPCM is equivalent to the Partial Credit Model.
\end{frame}

\section{Results}
\begin{frame}
  \frametitle{Create fake data}
\lstinputlisting{sim.R}
\end{frame}

\begin{frame}
\frametitle{Demographics}
Participants consisted of 200 fake
Charlottesville residents with
24\%
 having some training in meditation/yoga.

% latex table generated in R 2.15.0 by xtable 1.5-6 package
% Wed Apr 11 14:12:23 2012
\begin{table}[ht]
\begin{center}
\begin{tabular}{rr}
  \hline
 & Education \\ 
  \hline
high school &  54 \\ 
  college &  72 \\ 
  graduate &  74 \\ 
   \hline
\end{tabular}
\end{center}
\end{table}
Other demographic data were not fabricated.
\end{frame}

\begin{frame}
\includegraphics{gen/fig-003}
\end{frame}

\begin{frame}
Item fit sorted by Infit MSQ.
% latex table generated in R 2.15.0 by xtable 1.5-6 package
% Wed Apr 11 14:12:24 2012
\begin{table}[ht]
\begin{center}
\begin{tabular}{rrrrr}
  \hline
 & Outfit MSQ & Outfit t & Infit MSQ & Infit t \\ 
  \hline
notSense & 0.44 & -2.35 & 0.62 & -4.21 \\ 
  intention & 0.62 & -3.69 & 0.69 & -4.99 \\ 
  certain & 0.67 & -2.68 & 0.73 & -4.14 \\ 
  possMyself & 0.70 & -2.21 & 0.87 & -1.81 \\ 
  accident & 0.99 & -0.11 & 0.94 & -0.86 \\ 
  possAny & 1.17 & 0.90 & 1.05 & 0.61 \\ 
   \hline
\end{tabular}
\end{center}
\end{table}
Liberal acceptable range for MSQ 1.4-0.6
\end{frame}

\begin{frame}
\includegraphics{gen/fig-005}

Data for this item had the most variance (and underfit), simulating an
item that was not answered very consistently with respect to latent trait level.
\end{frame}

\begin{frame}
\includegraphics{gen/fig-006}

Data for this item was the most overfit for both infit and outfit.
\end{frame}

\begin{frame}
Person fit sorted by Infit MSQ.
% latex table generated in R 2.15.0 by xtable 1.5-6 package
% Wed Apr 11 14:12:26 2012
\begin{table}[ht]
\begin{center}
\begin{tabular}{rrrrr}
  \hline
 & Outfit MSQ & Outfit t & Infit MSQ & Infit t \\ 
  \hline
P3 & 0.10 & -0.46 & 0.14 & -2.01 \\ 
  P34 & 0.10 & -0.46 & 0.14 & -2.01 \\ 
  P36 & 0.10 & -0.46 & 0.14 & -2.01 \\ 
  P44 & 0.10 & -0.46 & 0.14 & -2.01 \\ 
  P72 & 0.10 & -0.46 & 0.14 & -2.01 \\ 
  P82 & 0.10 & -0.46 & 0.14 & -2.01 \\ 
  P121 & 1.77 & 1.90 & 2.29 & 2.97 \\ 
  P40 & 1.92 & 1.93 & 2.31 & 2.82 \\ 
  P83 & 2.27 & 1.64 & 2.67 & 2.91 \\ 
  P195 & 1.31 & 0.86 & 2.72 & 2.66 \\ 
  P172 & 4.03 & 3.71 & 2.78 & 3.39 \\ 
  P197 & 2.12 & 1.46 & 3.62 & 3.51 \\ 
   \hline
\end{tabular}
\end{center}
\end{table}
Liberal acceptable range for MSQ 1.4-0.6. Clearly there are some
crazy people here.
\end{frame}

\begin{frame}
A likely response pattern:
% latex table generated in R 2.15.0 by xtable 1.5-6 package
% Wed Apr 11 14:12:26 2012
\begin{table}[ht]
\begin{center}
\begin{tabular}{rrrrrrr}
  \hline
 & notSense & possAny & possMyself & accident & intention & certain \\ 
  \hline
3 &   2 &   2 &   2 &   2 &   1 &   1 \\ 
   \hline
\end{tabular}
\end{center}
\end{table}
Some unlikely response patterns:
% latex table generated in R 2.15.0 by xtable 1.5-6 package
% Wed Apr 11 14:12:26 2012
\begin{table}[ht]
\begin{center}
\begin{tabular}{rrrrrrr}
  \hline
 & notSense & possAny & possMyself & accident & intention & certain \\ 
  \hline
121 &   2 &   0 &   2 &   2 &   0 &   0 \\ 
  40 &   2 &   2 &   0 &   2 &   0 &   1 \\ 
  83 &   2 &   2 &   1 &   2 &   0 &   2 \\ 
  195 &   2 &   2 &   2 &   2 &   0 &   2 \\ 
  172 &   2 &   0 &   2 &   2 &   1 &   1 \\ 
  197 &   0 &   2 &   0 &   0 &   0 &   0 \\ 
   \hline
\end{tabular}
\end{center}
\end{table}\end{frame}

\begin{frame}
\includegraphics{gen/fig-010}

The information curve spans a reasonably wide range of trait scores.
\end{frame}

\begin{frame}
\includegraphics{gen/fig-011}
\end{frame}

\begin{frame}
\includegraphics{gen/fig-012}

Thresholds are out of order. This is bad. If the same thing is
seen in real data then there are too many response options or
the item needs refinement.
\end{frame}

\begin{frame}
Item fit sorted by Infit MSQ.
% latex table generated in R 2.15.0 by xtable 1.5-6 package
% Wed Apr 11 14:12:32 2012
\begin{table}[ht]
\begin{center}
\begin{tabular}{rrrrr}
  \hline
 & Outfit MSQ & Outfit t & Infit MSQ & Infit t \\ 
  \hline
cxSolving & 0.65 & -5.17 & 0.68 & -5.57 \\ 
  cxDaydream & 0.70 & -4.73 & 0.72 & -4.80 \\ 
  cxMeditate & 0.81 & -2.09 & 0.79 & -3.04 \\ 
  cxRelax & 0.98 & -0.19 & 0.97 & -0.45 \\ 
  cxSpirit & 1.08 & 1.03 & 0.99 & -0.12 \\ 
  cxReligion & 1.09 & 1.14 & 1.04 & 0.68 \\ 
  cxExercise & 1.15 & 1.92 & 1.06 & 1.02 \\ 
   \hline
\end{tabular}
\end{center}
\end{table}
Liberal acceptable range for MSQ 1.4-0.6
\end{frame}

\begin{frame}
Person fit sorted by Infit MSQ.
% latex table generated in R 2.15.0 by xtable 1.5-6 package
% Wed Apr 11 14:12:33 2012
\begin{table}[ht]
\begin{center}
\begin{tabular}{rrrrr}
  \hline
 & Outfit MSQ & Outfit t & Infit MSQ & Infit t \\ 
  \hline
P45 & 0.04 & -5.38 & 0.04 & -5.60 \\ 
  P180 & 0.08 & -3.81 & 0.06 & -4.25 \\ 
  P18 & 0.14 & -3.26 & 0.11 & -3.64 \\ 
  P175 & 0.17 & -4.04 & 0.17 & -4.00 \\ 
  P89 & 0.22 & -3.40 & 0.20 & -3.67 \\ 
  P6 & 0.22 & -3.32 & 0.25 & -3.20 \\ 
  P108 & 1.99 & 2.29 & 1.98 & 2.39 \\ 
  P96 & 2.15 & 2.23 & 2.00 & 2.03 \\ 
  P4 & 1.93 & 1.46 & 2.02 & 1.68 \\ 
  P82 & 1.82 & 1.85 & 2.14 & 2.49 \\ 
  P67 & 2.78 & 2.86 & 2.36 & 2.38 \\ 
  P116 & 3.07 & 3.74 & 2.52 & 3.11 \\ 
   \hline
\end{tabular}
\end{center}
\end{table}
Liberal acceptable range for MSQ 1.4-0.6. Again there are some
crazy people here.
\end{frame}

\begin{frame}
Some likely response pattern:
% latex table generated in R 2.15.0 by xtable 1.5-6 package
% Wed Apr 11 14:12:33 2012
\begin{table}[ht]
\begin{center}
{\tiny
\begin{tabular}{rrrrrrrr}
  \hline
 & cxReligion & cxSolving & cxSpirit & cxRelax & cxDaydream & cxMeditate & cxExercise \\ 
  \hline
45 &   2 &   3 &   3 &   2 &   1 &   3 &   2 \\ 
  180 &   3 &   3 &   3 &   3 &   2 &   3 &   3 \\ 
  18 &   1 &   2 &   1 &   1 &   0 &   2 &   1 \\ 
   \hline
\end{tabular}
}
\end{center}
\end{table}
Some unlikely response patterns:
% latex table generated in R 2.15.0 by xtable 1.5-6 package
% Wed Apr 11 14:12:33 2012
\begin{table}[ht]
\begin{center}
{\tiny
\begin{tabular}{rrrrrrrr}
  \hline
 & cxReligion & cxSolving & cxSpirit & cxRelax & cxDaydream & cxMeditate & cxExercise \\ 
  \hline
82 &   0 &   4 &   4 &   4 &   2 &   4 &   4 \\ 
  67 &   3 &   1 &   0 &   0 &   0 &   0 &   0 \\ 
  116 &   4 &   4 &   0 &   4 &   2 &   4 &   4 \\ 
   \hline
\end{tabular}
}
\end{center}
\end{table}\end{frame}

\begin{frame}
\includegraphics{gen/fig-017}

At least with this fake data, the information curve is somewhat
compressed between -2 and 2. If this happens with real data,
it may be worth considering other questions to better measure
the extremes of the trait.
\end{frame}

\section{Discussion}
\begin{frame}

The next step: collect real data.
\end{frame}

\bibliography{jpritikin} 
\bibliographystyle{apacite}

\end{document} 
