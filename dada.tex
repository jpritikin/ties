\documentclass[10pt,utf8x]{beamer}
\usefonttheme{serif}

\mode<presentation>
{
  \usetheme{CVille}
  \usecolortheme{cavs}
  \setbeamercovered{transparent}
  % or whatever (possibly just delete it)
}


\usepackage[english]{babel} % or whatever
\usepackage{multimedia}
\usepackage{apacite}
\usepackage{tikz}
\usepackage{color}
\usepackage{soul}

\newlength{\figurewidth} 
\figurewidth \textwidth  % This is for rectangular graphs 
\newlength{\figurewidthB} 
\figurewidthB .7\textwidth  % This is for square graphs 

\title{Is there a link between flow and meditation?}

\author{Joshua N. Pritikin}

\institute[University of Virginia] % (optional, but mostly needed)
{
  Department of Psychology\\
  University of Virginia
}

\date[DADA] % (optional, should be abbreviation of conference name)
{\footnotesize 19 Apr 2012}

\pgfdeclareimage[height=0.4cm]{university-logo}{logo.png}
\logo{\pgfuseimage{university-logo}}

\begin{document}

\begin{frame}
  \titlepage
\end{frame}

\setlength{\parskip}{2ex}

\section{Flow}
\begin{frame}
\frametitle{Flow a.k.a. Optimal Experience}
Flow is an intrinsically rewarding experience in which a person
is absorbed in the optimal performance of an activity. 

Flow can occur in diverse contexts \cite{csikszentmihalyi1991}
\begin{itemize}
\item Music performance
\item Reading
\item Writing
\item Mountain climbing
\item Japanese motorcycle gangs
\item etc
\end{itemize}
\end{frame}

\begin{frame}
\begin{figure}[tp]
{\scriptsize
\begin{tikzpicture}[>=latex,line join=bevel,scale=.7]
  \pgfsetlinewidth{1bp}
\input{gen/flow.tex}
\end{tikzpicture}
}
\label{fig:teleonomy-of-self}
\end{figure}
\end{frame}

\begin{frame}
\citeA[p.~103]{csikszentmihalyi1991} suggested
that yoga is "one of the oldest and most systematic methods of
producing the flow experience,"
in the sense of facilitating flow in other activities besides yoga/meditation
(cf. \citeNP[p.~31]{csikszentmihalyi1988}).

There is little outward commonality between stereotypical yogic
practices and the diverse activities in which flow has been found.

What aspect of yoga/meditation might facilitate flow?
\end{frame}

\begin{frame}
In the historical Indian yogic tradition, thoughtless awareness
(a.k.a. mental silence) is a primary goal of yoga/meditation.

The notion of thoughtless awareness is found in widely translated,
centuries old
books such as the Mahabharat, Upanishads, Yoga Sutras of Patanjali,
and Gyaneshwari \cite[pp.~93--94]{manocha2009}.

Thoughtless awareness is an experience in which one is
no longer is thinking and
no longer feels the compulsion to continuously engage in thought.
The experience occurs in a normal waking state;
thoughtless awareness is not similar to sleeping or unconsciousness.
\end{frame}

\begin{frame}
Can thoughtless awareness be a component of or itself a flow experience? 

\begin{itemize}
\item \textcolor{red}{Challenge/Skill} A naive thought suppression approach fails \cite{wegner2003}.
However, \citeA{manocha2000} reported that participants in his
Meditation Research Program ``consistently describe the
ability to achieve this experience'' (p.~1137).
Perhaps with the appropriate training,
the challenge of thoughtless awareness can be made navigable.

\item \textcolor{red}{Clear goal/Immediate feedback} Yes

\item \textcolor{red}{Concentration} Without engrossment
in thoughtless awareness, thinking is bound to resume.

\end{itemize}

Can meet the basic conditions of flow experience but ... ?
\end{frame}

\begin{frame}
Thoughtless awareness / mental silence has not been studied.

\begin{itemize}
\item PsycNET searches for ``thoughtless awareness'' or ``mental silence''
turn up nothing.
\item
In a collection of 13 scales related to spiritual and religious experience,
no attempt was made to measure ``thoughtless awareness'' \cite{fetzer1999}.
\item
In two recent reviews of the state of the research on meditation/mindfulness,
the possibility of thoughtless awareness or mental silence was not
even acknowledged (\citeNP{shapiro2009}; \citeNP{baer2011}).
\item
No glossary entry found in reference works on
positive psychology and the psychology of religion
(\citeNP{pargament1997}; \citeNP{peterson2004}; \citeNP{snyder2009}; \citeNP{sheldon2011}).
\item In comparison, extrasensory perception has attracted 2-3 orders of
magnitude more scientific studies \cite{lau2004}.
\end{itemize}

A potentially vital component of flow has not received
 any psychological scrutiny.
\end{frame}

\section{IRT}
\begin{frame}
  \frametitle{Item Response Theory -- Partial Credit Model}

\begin{displaymath}
P_{ix}(\theta) =
\frac{e^{\left[\ \displaystyle\sum_{j=0}^{x} (\theta - \delta_{ij})\right]}}
{\displaystyle\sum_{r=0}^{m_i}
e^{\left[\ \displaystyle\sum_{j=0}^{r}(\theta - \delta_{ij})\right]}}
\end{displaymath}

\begin{itemize}
\item $j$ is the person
\item $i$ is the item
\item $m_i$ is the number of choices for item $i$
\end{itemize}
\end{frame}

\section{Survey}
\begin{frame}
\begin{itemize}
\item Biographical information
\item Familiarity with thoughtless awareness
\item Thoughtless awareness in context
\item Supplemental questions
\end{itemize}
\end{frame}

\begin{frame}
  \frametitle{Biographical information}
Age?

Sex? Male/Female

Highest level of education attained?
\begin{itemize}
\item High school graduate
\item College Degree
\item Graduate Degree
\end{itemize}

Have you pursued any training in meditation? Yes/No

If so, what kind of meditation do you practice currently? (free text)
\end{frame}

\begin{frame}
  \frametitle{Familiarity with Thoughtless Awareness}
{\Large\ul{Introduction}}

Thoughtless awareness (a.k.a mental silence) is an experience in which
one is no longer thinking and no longer feels the compulsion to
continuously engage in thought. The experience occurs in a normal
waking state; thoughtless awareness is not similar to sleeping or
unconsciousness.

\end{frame}

\begin{frame}
  \frametitle{Familiarity with Thoughtless Awareness}

In the remainder of this survey, thoughtless awareness is abbreviated
as TA. Of interest are your recent experiences of TA (within the last
year or so).

On a 3-point Likert scale from \emph{agree} to \emph{disagree}, rate:

\begin{itemize}
\item The notion of TA does not make sense
\item I doubt that TA is possible for anybody
\item I doubt that TA is possible for myself
\item I have experienced TA accidentally
\item I have intentionally experienced TA
\item I am certain that TA is possible for anybody
\end{itemize}

\end{frame}

\begin{frame}
  \frametitle{Thoughtless Awareness in Context}

{\Large\ul{For religion}}

On a 3-point Likert scale from \emph{satisfactory} to
\emph{unsatisfactory}, rate:

\begin{itemize}
\item How satisfactory is \textcolor{red}{religion} with TA?
\item How satisfactory is \textcolor{red}{religion} without TA?
\end{itemize}

On a 5-point Likert scale from \emph{enough} to
\emph{not enough}, rate:

\begin{itemize}
\item How much TA do you experience in your \textcolor{red}{religion}?
\end{itemize}

\end{frame}

\begin{frame}
  \frametitle{Thoughtless Awareness in Context}

{\Large\ul{For \emph{context}}}

On a 3-point Likert scale from \emph{satisfactory} to
\emph{unsatisfactory}, rate:

\begin{itemize}
\item How satisfactory is \textcolor{red}{\emph{context}} with TA?
\item How satisfactory is \textcolor{red}{\emph{context}} without TA?
\end{itemize}

On a 5-point Likert scale from \emph{enough} to
\emph{not enough}, rate:

\begin{itemize}
\item How much TA do you experience in your \textcolor{red}{\emph{context}}?
\end{itemize}

For contexts: religion, spirituality, problem solving, relaxation,
daydreaming, and meditation.

\end{frame}

\bibliography{jpritikin} 
\bibliographystyle{apacite}

\end{document} 
