\documentclass[10pt,utf8x]{beamer}
\usefonttheme{serif}

\mode<presentation>
{
  \usetheme{CVille}
  \usecolortheme{cavs}
  \setbeamercovered{transparent}
  % or whatever (possibly just delete it)
}


\usepackage[english]{babel} % or whatever
\usepackage{multimedia}

%%%%%%%%%%%%%%%%%%%%%%%%%%%%%%%%%%%%%%%%%%%%%%%%%%%%%%%%%%%%%%%%% 
%\usepackage{geometry}
%\usepackage{multicols}
\usepackage{apacite}

\newlength{\figurewidth} 
\figurewidth \textwidth  % This is for rectangular graphs 
\newlength{\figurewidthB} 
\figurewidthB .7\textwidth  % This is for square graphs 


%%%%%%%%%%%%%%%%%%%%%%%%%%%%%%%%%%%%%%%%%%%%%%%%%%%%%%%%%%%%%%%%% 
%  Title and authors go here

\title[What are the proximate effects of mental silence?] % (optional, use only with long paper titles)
{What are the proximate effects of mental silence?}

%\subtitle{Include Only If Paper Has a Subtitle}

\author[Joshua N. Pritikin] % (optional, use only with lots of authors)
{Joshua N. Pritikin}
%{Steven M. Boker\inst{1} \and Another A. Author\inst{2}}
% - Give the names in the same order as the appear in the paper.
% - Use the \inst{?} command only if the authors have different
%   affiliation.

\institute[University of Virginia] % (optional, but mostly needed)
{
  Department of Psychology\\
  University of Virginia
}
%\institute[University of Notre Dame] % (optional, but mostly needed)
%{
%  \inst{1}%
%  Department of Psychology\\
%  University of Notre Dame
%  \and
%  \inst{2}%
%  Department of Theoretical Philosophy\\
%  University of Elsewhere}
% - Use the \inst command only if there are several affiliations.
% - Keep it simple, no one is interested in your street address.

\date[Nesselroade] % (optional, should be abbreviation of conference name)
{\footnotesize 27 Jan 2012}
% - Either use conference name or its abbreviation.
% - Not really informative to the audience, more for people (including
%   yourself) who are reading the slides online

\subject{Dynamical Systems}
% This is only inserted into the PDF information catalog. Can be left
% out. 



% If you have a file called "university-logo-filename.xxx", where xxx
% is a graphic format that can be processed by latex or pdflatex,
% resp., then you can add a logo as follows:

\pgfdeclareimage[height=0.4cm]{university-logo}{logo.png}
\logo{\pgfuseimage{university-logo}}



% Delete this, if you do not want the table of contents to pop up at
% the beginning of each subsection:
%\AtBeginSubsection[]
%{
%  \begin{frame}<beamer>
%    \frametitle{Outline}
%    \tableofcontents[currentsection,currentsubsection]
%  \end{frame}
%}


%%%%%%%%%%%%%%%%%%%%%%%%%%%%%%%%%%%%%%%%%%%%%%%%%%%%%%%%%%%%%%%%% 
% The document begins here 

\begin{document}

\begin{frame}
  \titlepage
\end{frame}

\section{Meditation techniques}
\begin{frame}
\frametitle{Meditation techniques}
Neither practitioners nor researchers have reached consensus
on what constitutes meditation practice (\citeNP{ospina2007}, p.~9).

\medskip
\citeA{ospina2007} identified 5 broad categories of practices:
\begin{itemize}
\item Mantra meditation (including Transcendental Meditation, Relaxation Response, and Clinically Standardized Meditation)
\item Mindfulness meditation (including Vispassana, Zen Buddhist meditation,
Mindfulness-based Stress Reduction, and Mindfulness-based Cognitive Therapy)
\item Yoga
\item Tai Chi
\item Qi Gong
\end{itemize}
\end{frame}

\begin{frame}
\frametitle{Meditation techniques}

Of interest here are meditative practices on the motionless side of
the moving-to-motionless continuum. Hence, no further consideration
will be given to Tai Chi or Qi Gong. Some types of yoga also emphasize
exercise and movement over awareness or insight.

\bigskip
How do the remaining techniques fall on an involvement-with-thought
continuum?

\end{frame}

\begin{frame}
\frametitle{Meditation techniques}

With respect to involvement-with-thought:
\begin{itemize}
\item Mantra meditation involves the repetition of thought forms.

\item Mindfulness may be defined as a state in which one passively observes
the ebb and flow of thoughts without getting involved with them.

\item Mental silence arises as a result of the meditator’s ability to not
only avoid initiating thoughts that may arise as a reaction to events
(as in mindfulness) but to completely eliminate even background mental
noise \cite{manocha2000}.
\end{itemize}

Meditation based on thoughtless awareness is the least studied.

\end{frame}

\section{Meditation or relaxation?}
\begin{frame}
\frametitle{Meditation or relaxation?}

Are any of these types of meditation significantly different from
rest or relaxation?

\medskip
One meta-analysis of 20 experiments using mantra meditation found
no strong evidence that meditation differed from rest \cite{holmes1984}.

\medskip
More recently, meta-analysis of meditation's effect on
hypertension (27 trials), other cardiovascular diseases (21 trials),
and substance abuse disorders (17 trials) were mostly inconclusive
in comparison to controls (\citeNP{ospina2007}, p.~4).
\end{frame}

\begin{frame}
\frametitle{Meditation or relaxation?}

Relaxation is not a bad goal, but is there something more to
meditation than relaxation?

\medskip
A mystery: Thoughtless awareness is an experience that
resists description (by definition).

\begin{itemize}
\item Contradicts DesCartes' ``I think therefore I am.'';
Consistent with \citeA{damasio1995}
\item Analogous to Heisenberg's uncertainty principle
(a macroscopic version). However, proximate effects can be investigated.
\end{itemize}

\medskip
Why has meditation attracted enduring popular interest?
Anecdotal evidences suggests something more than relaxation.

\end{frame}

\begin{frame}
\frametitle{Meditation or relaxation?}

Can mental silence meditation be distinguished from
rest/relaxation?

\medskip
One recent study found a decrease in skin temperature
in a mental silence meditation group compared to resting controls
\cite{manocha2010}.

\medskip
If this finding holds up, it suggests that mental silence meditation
is not entirely a state of reduced autonomic arousal
(i.e. para-sympathetic activation and sympathetic deactivation).
\end{frame}

\section{Proximate Effects}
\begin{frame}
\frametitle{Proximate effects}

Thoughtless awareness is not a new idea. It is mentioned in:
\begin{itemize}
\item the Upanishads, ancient spiritual writings from India
\item in Patanjali's Yoga Aphorisms, one of the first treatises on yoga and meditation
\item in writings by Gynaeshawara, a 12th century Indian mystic
\item Japanese Rinzai Zen; thoughtless awareness is hinted at using
 koans and described as ``satori''
\end{itemize}

Recently, there is renewed interest in thoughtless awareness
with the introduction of Sahaja yoga (a.k.a. Sahaja
meditation, Sahaja yoga meditation), a method developed
in the 20th century by Shri Mataji Nirmala Devi.
\end{frame}

\begin{frame}
\frametitle{Proximate effects}

Interventions using Sahaja meditation show promise:
\begin{itemize}
\item work stress, anxiety, and depressed mood \cite{manocha2011}
\item children with attention deficit-hyperactivity disorder \cite{harrison2004}
\item women in perimenopause \cite{manocha2007}
\item epilepsy \cite{panjwani2000}
\item asthma \cite{manocha2002}
\end{itemize}

\end{frame}

%%%%%%%%%%%%%%%%%%%%%%%%%%%%%%%%%%%%%%%%%%%%%%%%%%%%%%%%%%%%%%%%% 

\begin{frame}
  \frametitle{Ideas for Future Research}
\begin{itemize}
\item Survey on \emph{thoughtless awareness} and related context
\item Replicate \citeA{manocha2010}
\item ??
\end{itemize}

\end{frame}

\bibliography{jpritikin} 
\bibliographystyle{apacite}

\end{document} 

